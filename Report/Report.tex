\documentclass[a4paper,12pt]{article}
\usepackage{graphicx}
\usepackage{listings}
\usepackage{xcolor}
\usepackage{hyperref} 
\usepackage{caption} 
\usepackage{subcaption}
\usepackage{fancyhdr} 
\usepackage{geometry}
\usepackage{titling}
\usepackage{tocbibind}
\usepackage{float} 
\usepackage{array}
\usepackage{longtable}
\usepackage[dvipsnames]{xcolor}

\geometry{top=3cm, bottom=3cm, left=2.5cm, right=2.5cm}

\begin{document}
\thispagestyle{empty} 

\vspace*{1cm} 

\begin{center}
    {\Huge\textbf {Chess Project}}  
    \vspace{1cm}  

    \begin{figure}[h!]
        \centering
        \includegraphics[width=0.3\textwidth]{Images/logo.jpg}
    \end{figure}
    
    {\large Session 2023-2027}\\[0.2cm]
    \vspace{0.7cm}
    {\Large\textbf {Submitted By: }}\\[0.6cm]
    {\large {Mohammad Bilal \space\space\space 2023-CS-168}}\\[0.6cm]
    {\Large\textbf {Supervised By: }}\\[0.6cm]
    {\large {Mr. Nazeef Ul Haq }}\\[0.4cm]
    {\large {Mr. Waseem }}\\[0.6cm]
    {\Large\textbf {Course: } CSC-200}\\[1.2cm]
    {\LARGE \textbf{ Department of Computer Science}}\\[0.7cm]
    {\Huge \textbf{University of Engineering and}}\\[0.5cm]
    {\Huge \textbf{Technology, Lahore}}\\[0.5cm]
    \vfill
\end{center}

\thispagestyle{empty} 

% all the contents table, figures and tables must have no text on heder footer 

\renewcommand{\contentsname}{Table of Contents}

\tableofcontents

\newpage

\thispagestyle{empty}

\listoffigures

\newpage

\thispagestyle{empty}

\listoftables

\newpage

\pagestyle{fancy}
\fancyhf{}
\fancyhead[L]{\textbf{Chess Project}}
\fancyfoot[R]{{\thepage}}
\fancyfoot[L]{{MOHAMMAD BILAL}}

\section{Project Overview}

\subsection{Description}
This project is a Chess Game developed using the WPF framework in C\#. It leverages the .NET Framework version v4.8 and provides an engaging platform for both multiplayer and single-player modes. The single-player mode offers three difficulty levels: easy, medium, and hard. Additionally, the game adheres strictly to FIDE rules, includes time controls (1m, 3m, 5m, and 10m), and supports advanced features like Undo functionality, offering a draw, and move validation. 

\subsection{Motivation}
The motivation for creating this project stemmed from two key factors:
\begin{itemize}
    \item A desire to learn and understand the practical use of data structures in building software applications.
    \item A personal admiration for the game of Chess and an eagerness to develop my own version of it with modern features and functionalities.
\end{itemize}

\subsection{Objectives}
The objectives of the project include the following.
\begin{itemize}
    \item Design and implement a chess game that adheres to the FIDE rules.
    \item To integrate both multiplayer and Player vs. Computer modes with varying difficulty levels.
    \item Demonstrate the use of data structures such as arrays, lists, linked lists, stacks, and graphs in a real-world project.
    \item To enhance understanding of WPF and the.NET framework.
\end{itemize}

\subsection{Target Audience}
This project aims to:
\begin{itemize}
    \item Beginners and intermediate programmers who wish to explore the practical application of data structures.
    \item Developers interested in creating games using WPF and the.NET Framework.
\end{itemize}

\subsection{Features}
\begin{longtable}{|p{0.25\textwidth}|p{0.7\textwidth}|}
    \caption{Game Features}
    \hline
    \textbf{Feature} & \textbf{Description} \\
    \hline
    Multiplayer Mode & Allows two human players to play against each other. \\
    \hline
    Notation & Proper Display of Moves Notation for ease of users. \\
    \hline
    Player vs. Computer Mode & Enables a single player to compete against the AI with three difficulty levels: easy, medium, and hard. \\
    \hline
    Time Controls & Provides four time control options: 1 minute, 3 minutes, 5 minutes, and 10 minutes. \\
    \hline
    FIDE-Compliant Rules & Adheres to official Chess rules, including castling, en passant, and pawn promotion. \\
    \hline
    Undo Functionality & Allows players to undo their last move. \\
    \hline
    Offer Draw & Players can offer a draw during the game. \\
    \hline
    Resign & Players can resign during the game. \\
    \hline
\end{longtable}

\subsection{Operational Details}
\begin{longtable}{|p{0.25\textwidth}|p{0.7\textwidth}|}
    \caption{Technology Stack}
    \hline
    Framework & .NET Framework v4.8. \\
    \hline
    UI Framework & WPF (Windows Presentation Foundation)   \\
    \hline
    Language & C\# \\
    \hline
    IDE & Visual Studio 2022 Community Edition \\
    \hline
\end{longtable}

\section{Use Cases}

\subsection{Landing Page}

\begin{longtable}{|m{0.25\textwidth}|m{0.7\textwidth}|}
    \caption{Landing Page} \\
    \hline
    Name & Landing page \\
    \hline
    Actor & Player \\
    \hline
    Description & It displays the options that the player can explore in this game. \\ 
    \hline
    \centering UI & 
    \begin{center}
        \includegraphics[height=2.7in]{Images/Use Cases/landingPage.png}
    \end{center} \\ 
    \hline
\end{longtable}

\subsection{Multi-Player Select Options}

\begin{longtable}{|m{0.25\textwidth}|m{0.7\textwidth}|}
    \caption{Multi-Player Select Options} \\
    \hline
    Name & Multi-player pop-up \\
    \hline
    Actor & Player \\
    \hline
    Description & It displays the time controls and the color that the player can select. \\ 
    \hline
    \centering UI & 
    \begin{center}
        \includegraphics[height=2.7in]{Images/Use Cases/multiplayerSelectOptions.png}
    \end{center} \\ 
    \hline
\end{longtable}

\subsection{Multi-Player Page}

\begin{longtable}{|m{0.25\textwidth}|m{0.7\textwidth}|}
    \caption{Multi-player Page} \\
    \hline
    Name & Game page \\
    \hline
    Actor & Player \\
    \hline
    Description & It is the main page where the game is played. It displays the board, moves, dead pieces and the additional buttons for better experience \\ 
    \hline
    \centering UI & 
    \begin{center}
        \includegraphics[height=2.7in]{Images/Use Cases/multiplayerPage.png}
    \end{center} \\ 
    \hline
\end{longtable}

\subsection{Promotion Select Options}

\begin{longtable}{|m{0.25\textwidth}|m{0.7\textwidth}|}
    \caption{Promotion Select Options} \\
    \hline
    Name & Promotion selection page \\
    \hline
    Actor & Player \\
    \hline
    Description & It displays the options (Queen, Knight, Bishop and Rook) that your pawn can promote to on reaching enemies last rank.\\ 
    \hline
    \centering UI & 
    \begin{center}
        \includegraphics[height=2.3in]{Images/Use Cases/promotionSelectOptions.png}
    \end{center} \\ 
    \hline
\end{longtable}

\subsection{Multi-Player Game Progress}

\begin{longtable}{|m{0.25\textwidth}|m{0.7\textwidth}|}
    \caption{Multi-Player Game Progress} \\
    \hline
    Name & Game page \\
    \hline
    Actor & Player \\
    \hline
    Description & It displays current progress of the game. On top right side are the moves notations, and below are both player's dead pieces\\ 
    \hline
    \centering UI & 
    \begin{center}
        \includegraphics[height=2.7in]{Images/Use Cases/multiplayerProgress.png}
    \end{center} \\ 
    \hline
\end{longtable}

\subsection{Vs Computer Select Options}

\begin{longtable}{|m{0.25\textwidth}|m{0.7\textwidth}|}
    \caption{Vs Computer Select Options} \\
    \hline
    Name & Vs Computer Select Options \\
    \hline
    Actor & Player \\
    \hline
    Description & It displays the time controls, difficulty levels and the color that the player can select.  \\ 
    \hline
    \centering UI & 
    \begin{center}
        \includegraphics[height=2.7in]{Images/Use Cases/vsComputerSelectOptions.png}
    \end{center} \\ 
    \hline
\end{longtable}

\subsection{Vs Computer Progress Page}

\begin{longtable}{|m{0.25\textwidth}|m{0.7\textwidth}|}
    \caption{Vs Computer Progress Page} \\
    \hline
    Name & Vs Computer Progress page \\
    \hline
    Actor & Player \\
    \hline
    Description & It displays current progress of the game. We can clearly see that the time of Black(Computer) is not used at all which shows it's fast move selection. \\ 
    \hline
    \centering UI & 
    \begin{center}
        \includegraphics[height=2.7in]{Images/Use Cases/vsComputerProgress.png}
    \end{center} \\ 
    \hline
\end{longtable}

\subsection{About Page}

\begin{longtable}{|m{0.25\textwidth}|m{0.7\textwidth}|}
    \caption{About Page} \\
    \hline
    Name & About page \\
    \hline
    Actor & Player \\
    \hline
    Description & It displays the develepor info, project's features and developer's social links. \\ 
    \hline
    \centering UI & 
    \begin{center}
        \includegraphics[height=2.7in]{Images/Use Cases/aboutPage.png}
    \end{center} \\ 
    \hline
\end{longtable}

\section{Test Cases}

\subsection{Test Case 1: Knight Valid Moves}
\textbf{Description}: Ensure that White's Queenside Knight can move to valid squares.\\
\textbf{Expected Outcome}: White's Queenside Knight can move to a3 and c3 from b1.\\
\textbf{Result}: Requirement Satisfied.

\begin{figure}[H]
    \centering
    \includegraphics[width=0.7\linewidth]{Images/Test Cases/testCase1Img1.png}
    \caption{On Clicking on the Knight}
    \label{fig:BeforeKnightMove}
\end{figure}

\begin{figure}[H]
    \centering
    \includegraphics[width=0.7\linewidth]{Images/Test Cases/testCase1Img2.png}
    \caption{On Clicking on the c3 square}
    \label{fig:AfterKnightMove}
\end{figure}

\subsection{Test Case 2: En Passant}
\textbf{Description}: Ensure that Black can perform En Passant capture on White's pawn.\\
\textbf{Expected Outcome}: Black can capture White's pawn on g3 using En Passant.\\
\textbf{Result}: Requirement Satisfied.

\begin{figure}[H]
    \centering
    \includegraphics[width=0.7\linewidth]{Images/Test Cases/testCase2Img1.png}
    \caption{Moving White's Pawn}
    \label{fig:WhiteMove}
\end{figure}

\begin{figure}[H]
    \centering
    \includegraphics[width=0.7\linewidth]{Images/Test Cases/testCase2Img2.png}
    \caption{Before En Passant Capture}
    \label{fig:BeforeEnPassant}
\end{figure}

\begin{figure}[H]
    \centering
    \includegraphics[width=0.7\linewidth]{Images/Test Cases/testCase2Img3.png}
    \caption{After En Passant Capture}
    \label{fig:AfterEnPassant}
\end{figure}

\subsection{Test Case 3: Check}
\textbf{Description}: Ensure that the game detects when a king is in check.\\
\textbf{Expected Outcome}: The game correctly highlights the check condition when the White King is threatened by the Black's Dark-Square Bishop.\\
\textbf{Result}: Requirement Satisfied.

\begin{figure}[H]
    \centering
    \includegraphics[width=0.7\linewidth]{Images/Test Cases/testCase3Img1.png}
    \caption{Before Check}
    \label{fig:BeforeCheck}
\end{figure}

\begin{figure}[H]
    \centering
    \includegraphics[width=0.7\linewidth]{Images/Test Cases/testCase3Img2.png}
    \caption{After Check}
    \label{fig:AfterCheck}
\end{figure}

\subsection{Test Case 4: En Passant Undo}
\textbf{Description}: Ensure that an En Passant move can be undone correctly.\\
\textbf{Expected Outcome}: En Passant capture can be undone, and the pawn returns to its original position.\\
\textbf{Result}: Requirement Satisfied.

\begin{figure}[H]
    \centering
    \includegraphics[width=0.7\linewidth]{Images/Test Cases/testCase4Img1.png}
    \caption{Before Undoing En Passant}
    \label{fig:BeforeEnPassantUndo}
\end{figure}

\begin{figure}[H]
    \centering
    \includegraphics[width=0.7\linewidth]{Images/Test Cases/testCase4Img2.png}
    \caption{After Undoing En Passant}
    \label{fig:AfterEnPassantUndo}
\end{figure}

\subsection{Test Case 5(Part 1): Notation for Promotion}
\textbf{Description}: Ensure that the promotion notation is displayed correctly when a pawn reaches the 8th rank.\\
\textbf{Expected Outcome}: The game displays the correct notation for the promotion of a White pawn to a Queen.\\
\textbf{Result}: Requirement Satisfied.

\begin{figure}[H]
    \centering
    \includegraphics[width=0.7\linewidth]{Images/Test Cases/testCase5Part1Img1.png}
    \caption{Before Promotion Notation}
    \label{fig:BeforePromotionNotation}
\end{figure}

\begin{figure}[H]
    \centering
    \includegraphics[width=0.7\linewidth]{Images/Test Cases/testCase5Part1Img2.png}
    \caption{After Promotion Notation}
    \label{fig:AfterPromotionNotation}
\end{figure}

\subsection{Test Case 5(Part 2): Notation for Pieces Attacking Same Block}
\textbf{Description}: Ensure that pieces attacking the same block are highlighted correctly.\\
\textbf{Expected Outcome}: The game correctly highlights the squares where multiple pieces are attacking.\\
\textbf{Result}: Requirement Satisfied.

\begin{figure}[H]
    \centering
    \includegraphics[width=0.7\linewidth]{Images/Test Cases/testCase5Part2Img1.png}
    \caption{Before Pieces Attacking Same Block}
    \label{fig:BeforeAttackingSameBlock}
\end{figure}

\begin{figure}[H]
    \centering
    \includegraphics[width=0.7\linewidth]{Images/Test Cases/testCase5Part2Img2.png}
    \caption{After Pieces Attacking Same Block}
    \label{fig:AfterAttackingSameBlock}
\end{figure}

\subsection{Test Case 6: Castling}
\textbf{Description}: Ensure that Black can perform short castle with the king and rook.\\
\textbf{Expected Outcome}: Black can castle by moving the King from e8 to g8 and the Rook from h8 to f8.\\
\textbf{Result}: Requirement Satisfied.

\begin{figure}[H]
    \centering
    \includegraphics[width=0.7\linewidth]{Images/Test Cases/testCase6Img1.png}
    \caption{Before Castling}
    \label{fig:BeforeCastling}
\end{figure}

\begin{figure}[H]
    \centering
    \includegraphics[width=0.7\linewidth]{Images/Test Cases/testCase6Img2.png}
    \caption{After Castling}
    \label{fig:AfterCastling}
\end{figure}

\subsection{Test Case 7: Scholar's Mate}
\textbf{Description}: Ensure that Scholar's Mate can be achieved by White in 4 moves.\\
\textbf{Expected Outcome}: White delivers Scholar's Mate by moving the Queen to f3 and checkmates the Black King.\\
\textbf{Result}: Requirement Satisfied.

\begin{figure}[H]
    \centering
    \includegraphics[width=0.7\linewidth]{Images/Test Cases/testCase7Img1.png}
    \caption{Before Scholar's Mate}
    \label{fig:BeforeScholarsMate}
\end{figure}

\begin{figure}[H]
    \centering
    \includegraphics[width=0.7\linewidth]{Images/Test Cases/testCase7Img2.png}
    \caption{After Scholar's Mate}
    \label{fig:AfterScholarsMate}
\end{figure}

\subsection{Test Case 8: Check by Castling}
\textbf{Description}: Ensure that Check by castling is possible.\\
\textbf{Expected Outcome}: The game detects that castling for an open d file leads to check by rook of opponent king.\\
\textbf{Result}: Requirement Satisfied.

\begin{figure}[H]
    \centering
    \includegraphics[width=0.7\linewidth]{Images/Test Cases/testCase8Img1.png}
    \caption{Before Check by Castling}
    \label{fig:BeforeCheckByCastling}
\end{figure}

\begin{figure}[H]
    \centering
    \includegraphics[width=0.7\linewidth]{Images/Test Cases/testCase8Img2.png}
    \caption{After Check by Castling}
    \label{fig:AfterCheckByCastling}
\end{figure}

\subsection{Test Case 9: Promotion Check}
\textbf{Description}: Ensure that a promotion results in check if the opponent's king is in danger.\\
\textbf{Expected Outcome}: The promotion results in check if the opponent's king is in check.\\
\textbf{Result}: Requirement Satisfied.

\begin{figure}[H]
    \centering
    \includegraphics[width=0.7\linewidth]{Images/Test Cases/testCase9Img1.png}
    \caption{Before Promotion Check}
    \label{fig:BeforePromotionCheck}
\end{figure}

\begin{figure}[H]
    \centering
    \includegraphics[width=0.7\linewidth]{Images/Test Cases/testCase9Img2.png}
    \caption{After Promotion Check}
    \label{fig:AfterPromotionCheck}
\end{figure}

\subsection{Test Case 10: Computer Move}
\textbf{Description}: Ensure that the computer makes a valid move according to the chosen difficulty.\\
\textbf{Expected Outcome}: The computer makes a valid move on the board for each difficulty level.\\
\textbf{Result}: Requirement Satisfied.

\begin{figure}[H]
    \centering
    \includegraphics[width=0.7\linewidth]{Images/Test Cases/testCase10Img1.png}
    \caption{Before Computer Move}
    \label{fig:BeforeComputerMove}
\end{figure}

\begin{figure}[H]
    \centering
    \includegraphics[width=0.7\linewidth]{Images/Test Cases/testCase10Img2.png}
    \caption{After Computer Move}
    \label{fig:AfterComputerMove}
\end{figure}

\end{document}
